\documentclass[journal,12pt,twocolumn]{IEEEtran}
\usepackage[compact]{titlesec}
\usepackage{setspace}
\usepackage{gensymb}
\singlespacing
\usepackage[cmex10]{amsmath}
\usepackage{amsthm}
\usepackage{mathrsfs}
\usepackage{txfonts}
\usepackage{stfloats}
\usepackage{bm}
\usepackage{cite}
\usepackage{cases}
\usepackage{subfig}
\usepackage{longtable}
\usepackage{multirow}
\usepackage{enumitem}
\usepackage{mathtools}
\usepackage{steinmetz}
\usepackage{tikz}
\usepackage{circuitikz}
\usepackage{verbatim}
\usepackage{tfrupee}
\usepackage[breaklinks=true]{hyperref}
\usepackage{tkz-euclide}

\usetikzlibrary{calc,math}
\usepackage{listings}
    \usepackage{color}                                            %%
    \usepackage{array}                                            %%
    \usepackage{longtable}                                        %%
    \usepackage{calc}                                             %%
    \usepackage{multirow}                                         %%
    \usepackage{hhline}                                           %%
    \usepackage{ifthen}                                           %%

    \usepackage{lscape}     
\usepackage{multicol}
\usepackage{chngcntr}

\DeclareMathOperator*{\Res}{Res}
\renewcommand\thesection{\arabic{section}}
\renewcommand\thesubsection{\thesection.\arabic{subsection}}
\renewcommand\thesubsubsection{\thesubsection.\arabic{subsubsection}}

\renewcommand\thesectiondis{\arabic{section}}
\renewcommand\thesubsectiondis{\thesectiondis.\arabic{subsection}}
\renewcommand\thesubsubsectiondis{\thesubsectiondis.\arabic{subsubsection}}

\hyphenation{op-tical net-works semi-conduc-tor}
\def\inputGnumericTable{}                                 %%

\lstset{
frame=single, 
breaklines=true,
columns=fullflexible
}

\begin{document}

\newtheorem{theorem}{Theorem}[section]
\newtheorem{problem}{Problem}
\newtheorem{proposition}{Proposition}[section]
\newtheorem{lemma}{Lemma}[section]
\newtheorem{corollary}[theorem]{Corollary}
\newtheorem{example}{Example}[section]
\newtheorem{definition}[problem]{Definition}
\newcommand{\BEQA}{\begin{eqnarray}}
\newcommand{\EEQA}{\end{eqnarray}}
\newcommand{\define}{\stackrel{\triangle}{=}}
\bibliographystyle{IEEEtran}

\providecommand{\mbf}{\mathbf}
\providecommand{\pr}[1]{\ensuremath{\Pr\left(#1\right)}}
\providecommand{\qfunc}[1]{\ensuremath{Q\left(#1\right)}}
\providecommand{\sbrak}[1]{\ensuremath{{}\left[#1\right]}}
\providecommand{\lsbrak}[1]{\ensuremath{{}\left[#1\right.}}
\providecommand{\rsbrak}[1]{\ensuremath{{}\left.#1\right]}}
\providecommand{\brak}[1]{\ensuremath{\left(#1\right)}}
\providecommand{\lbrak}[1]{\ensuremath{\left(#1\right.}}
\providecommand{\rbrak}[1]{\ensuremath{\left.#1\right)}}
\providecommand{\cbrak}[1]{\ensuremath{\left\{#1\right\}}}
\providecommand{\lcbrak}[1]{\ensuremath{\left\{#1\right.}}
\providecommand{\rcbrak}[1]{\ensuremath{\left.#1\right\}}}
\theoremstyle{remark}
\newtheorem{rem}{Remark}
\newcommand{\sgn}{\mathop{\mathrm{sgn}}}
\providecommand{\abs}[1]{\left\vert#1\right\vert}
\providecommand{\res}[1]{\Res\displaylimits_{#1}} 
\providecommand{\norm}[1]{\left\lVert#1\right\rVert}

\providecommand{\mtx}[1]{\mathbf{#1}}
\providecommand{\mean}[1]{E\left[ #1 \right]}
\providecommand{\fourier}{\overset{\mathcal{F}}{ \rightleftharpoons}}

\providecommand{\system}{\overset{\mathcal{H}}{ \longleftrightarrow}}
\newcommand{\solution}{\noindent \textbf{Solution: }}
\newcommand{\cosec}{\,\text{cosec}\,}
\providecommand{\dec}[2]{\ensuremath{\overset{#1}{\underset{#2}{\gtrless}}}}
\newcommand{\myvec}[1]{\ensuremath{\begin{pmatrix}#1\end{pmatrix}}}
\newcommand{\mydet}[1]{\ensuremath{\begin{vmatrix}#1\end{vmatrix}}}
\numberwithin{equation}{subsection}

\makeatletter
\@addtoreset{figure}{problem}
\makeatother
\let\StandardTheFigure\thefigure
\let\vec\mathbf

\renewcommand{\thefigure}{\theproblem}

\def\putbox#1#2#3{\makebox[0in][l]{\makebox[#1][l]{}\raisebox{\baselineskip}[0in][0in]{\raisebox{#2}[0in][0in]{#3}}}}
     \def\rightbox#1{\makebox[0in][r]{#1}}
     \def\centbox#1{\makebox[0in]{#1}}
     \def\topbox#1{\raisebox{-\baselineskip}[0in][0in]{#1}}
     \def\midbox#1{\raisebox{-0.5\baselineskip}[0in][0in]{#1}}
\vspace{3cm}
\title{Assignment-10}
\author{Vipul Kumar Malik}

\date{\today}

\maketitle
\newpage
\bigskip
\renewcommand{\thefigure}{\theenumi}
\renewcommand{\thetable}{\theenumi}

\begin{abstract}
This document explains the relationship between the rank of matrix and the solution of the linear system of equations.
\end{abstract}
Download all python codes from 
\begin{lstlisting}
https://github.com/vipulmalik8569/MT-EE5609
\end{lstlisting}
and latex-tikz codes from 
\begin{lstlisting}
https://github.com/vipulmalik8569/MT-EE5609
\end{lstlisting}
\section{Problem}
Let $\vec A$ be an $m$ x $n$ matrix with rank $n$ and real entries. Which of the following statements are correct?
\begin{enumerate}
    \item $\vec{A}\vec{x}=\vec{b}$ has a solution for any $\vec{b}$.
    \item $\vec{A}\vec{x}=0$ does not have a solution.
    \item If $\vec{A}\vec{x}=\vec{b}$ has a solution then it is unique.
    \item $\vec{y}^T\vec{A}=0$ for some non zero vector $\vec{y}$.
\end{enumerate}
\section{solution}
\subsection{Option 1}
A solution exist only if $\vec{b}$ lies in the column space of $\vec{A}$. Let us take an example 
\begin{align}
    \vec{A}\vec{x}&=\vec{b}\\
    \myvec{1&0\\0&1\\0&0}\vec{x}&=\myvec{1\\2\\3}
\end{align}
Here the system has no solution because the column space of matrix $\vec{A}$ is x-y plane and $\vec{b}$ lies outside it.


Thus, option 1 is incorrect.
\subsection{Option 2}
For the system $\vec{A}\vec{x}=0$, $\vec{x}=0$ is always the solution.


Therefore, option 2 is incorrect.
\subsection{Option 3}
There are two possibilities of matrices with dimension $m$ x $n$ and rank $n$ : 
\begin{enumerate}
    \item Rectangular matrix with $m>n$.
    \item Square matrix with $m=n$.
\end{enumerate}
Therefore there is no possibility for the system to have infinitely many solutins.

Hence, option 3 is correct.
\subsection{Option 4}
Let us consider an example with $\vec{u}$ and $\vec{v}$ as the column vectors of $\vec{A}$ : 
\begin{align}
    \vec{y}^T\myvec{\vec{u}&\vec{v}}&=0\\
    \myvec{\vec{y}^T\vec{u}&\vec{y}^T\vec{v}}&=0
\end{align}
This is possible only when vectors $\vec{u}$, $\vec{v}$ and $\vec{y}$ are orthogonal to each other.\\
Example :
\begin{align}
    \myvec{0&0&1}\myvec{1&0\\0&1\\0&0}=\myvec{0&0}\\
    \vec{u}=\myvec{1\\0\\0}\quad\vec{v}=\myvec{0\\1\\0}\quad\vec{y}=\myvec{0\\0\\1}
\end{align}
Hence, option 4 is correct.
\end{document}
