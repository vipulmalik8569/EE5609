\documentclass[journal,12pt,twocolumn]{IEEEtran}
\usepackage{setspace}
\usepackage{gensymb}
\singlespacing
\usepackage[cmex10]{amsmath}
\usepackage{amsthm}
\usepackage{mathrsfs}
\usepackage{txfonts}
\usepackage{stfloats}
\usepackage{bm}
\usepackage{cite}
\usepackage{cases}
\usepackage{subfig}
\usepackage{longtable}
\usepackage{multirow}
\usepackage{enumitem}
\usepackage{mathtools}
\usepackage{steinmetz}
\usepackage{tikz}
\usepackage{circuitikz}
\usepackage{verbatim}
\usepackage{tfrupee}
\usepackage[breaklinks=true]{hyperref}
\usepackage{tkz-euclide}

\usetikzlibrary{calc,math}
\usepackage{listings}
    \usepackage{color}                                            %%
    \usepackage{array}                                            %%
    \usepackage{longtable}                                        %%
    \usepackage{calc}                                             %%
    \usepackage{multirow}                                         %%
    \usepackage{hhline}                                           %%
    \usepackage{ifthen}                                           %%

    \usepackage{lscape}     
\usepackage{multicol}
\usepackage{chngcntr}

\DeclareMathOperator*{\Res}{Res}
\renewcommand\thesection{\arabic{section}}
\renewcommand\thesubsection{\thesection.\arabic{subsection}}
\renewcommand\thesubsubsection{\thesubsection.\arabic{subsubsection}}

\renewcommand\thesectiondis{\arabic{section}}
\renewcommand\thesubsectiondis{\thesectiondis.\arabic{subsection}}
\renewcommand\thesubsubsectiondis{\thesubsectiondis.\arabic{subsubsection}}

\hyphenation{op-tical net-works semi-conduc-tor}
\def\inputGnumericTable{}                                 %%

\lstset{
frame=single, 
breaklines=true,
columns=fullflexible
}

\begin{document}

\newtheorem{theorem}{Theorem}[section]
\newtheorem{problem}{Problem}
\newtheorem{proposition}{Proposition}[section]
\newtheorem{lemma}{Lemma}[section]
\newtheorem{corollary}[theorem]{Corollary}
\newtheorem{example}{Example}[section]
\newtheorem{definition}[problem]{Definition}
\newcommand{\BEQA}{\begin{eqnarray}}
\newcommand{\EEQA}{\end{eqnarray}}
\newcommand{\define}{\stackrel{\triangle}{=}}
\bibliographystyle{IEEEtran}

\providecommand{\mbf}{\mathbf}
\providecommand{\pr}[1]{\ensuremath{\Pr\left(#1\right)}}
\providecommand{\qfunc}[1]{\ensuremath{Q\left(#1\right)}}
\providecommand{\sbrak}[1]{\ensuremath{{}\left[#1\right]}}
\providecommand{\lsbrak}[1]{\ensuremath{{}\left[#1\right.}}
\providecommand{\rsbrak}[1]{\ensuremath{{}\left.#1\right]}}
\providecommand{\brak}[1]{\ensuremath{\left(#1\right)}}
\providecommand{\lbrak}[1]{\ensuremath{\left(#1\right.}}
\providecommand{\rbrak}[1]{\ensuremath{\left.#1\right)}}
\providecommand{\cbrak}[1]{\ensuremath{\left\{#1\right\}}}
\providecommand{\lcbrak}[1]{\ensuremath{\left\{#1\right.}}
\providecommand{\rcbrak}[1]{\ensuremath{\left.#1\right\}}}
\theoremstyle{remark}
\newtheorem{rem}{Remark}
\newcommand{\sgn}{\mathop{\mathrm{sgn}}}
\providecommand{\abs}[1]{\left\vert#1\right\vert}
\providecommand{\res}[1]{\Res\displaylimits_{#1}} 
\providecommand{\norm}[1]{\left\lVert#1\right\rVert}

\providecommand{\mtx}[1]{\mathbf{#1}}
\providecommand{\mean}[1]{E\left[ #1 \right]}
\providecommand{\fourier}{\overset{\mathcal{F}}{ \rightleftharpoons}}

\providecommand{\system}{\overset{\mathcal{H}}{ \longleftrightarrow}}
\newcommand{\solution}{\noindent \textbf{Solution: }}
\newcommand{\cosec}{\,\text{cosec}\,}
\providecommand{\dec}[2]{\ensuremath{\overset{#1}{\underset{#2}{\gtrless}}}}
\newcommand{\myvec}[1]{\ensuremath{\begin{pmatrix}#1\end{pmatrix}}}
\newcommand{\mydet}[1]{\ensuremath{\begin{vmatrix}#1\end{vmatrix}}}
\numberwithin{equation}{subsection}

\makeatletter
\@addtoreset{figure}{problem}
\makeatother
\let\StandardTheFigure\thefigure
\let\vec\mathbf

\renewcommand{\thefigure}{\theproblem}

\def\putbox#1#2#3{\makebox[0in][l]{\makebox[#1][l]{}\raisebox{\baselineskip}[0in][0in]{\raisebox{#2}[0in][0in]{#3}}}}
     \def\rightbox#1{\makebox[0in][r]{#1}}
     \def\centbox#1{\makebox[0in]{#1}}
     \def\topbox#1{\raisebox{-\baselineskip}[0in][0in]{#1}}
     \def\midbox#1{\raisebox{-0.5\baselineskip}[0in][0in]{#1}}
\vspace{3cm}
\title{Assignment-3}
\author{Vipul Kumar Malik \\ AI20MTECH14006}

\date{\today}

\maketitle
\newpage
\bigskip
\renewcommand{\thefigure}{\theenumi}
\renewcommand{\thetable}{\theenumi}

\begin{abstract}
This document explains the concept of balancing the chemical equations using linear algebra.
\end{abstract}
Download all python codes from 
\begin{lstlisting}
https://github.com/vipulmalik8569/MT-EE5609
\end{lstlisting}
and latex-tikz codes from 
\begin{lstlisting}
https://github.com/vipulmalik8569/MT-EE5609
\end{lstlisting}
\section{\textbf{Problem}}
Write the balanced chemical equations for the following reaction :
\begin{align}
 BaCl_2 + K_2SO_4 \rightarrow BaSO_4 + KCl \label{eq:1}   
\end{align}
\section{\textbf{Solution}}
We know that the number of atoms of each element remains the
same, before and after a chemical reaction.

Equation \eqref{eq:1} can be written as:
\begin{align}
x_1BaCl_2 + x_2K_2SO_4 \rightarrow x_3BaSO_4 + x_4KCl\label{eq:2}
\end{align}
Element wise contribution in forming the respective chemical compound can be written in the form of equation as :
\begin{align}
Ba : x_1 + 0x_2 - x_3 - 0x_4 = 0\\
Cl : 2x_1 + 0x_2 - 0x_3 - 1x_4 = 0\\
K  : 0x_1 + 2x_2 - 0x_3 - 1x_4 = 0\\
S  : 0x_1 + 1x_2 - 1x_3 - 0x_4 = 0\\
O  : 0x_1 + 4x_2 - 4x_3 - 0x_4 = 0
\end{align}
In matrix form this can be written as:
\begin{align}
A\vec{x}&=0\\
  \myvec{1&0&-1&0\\2&0&0&-1\\0&2&0&-1\\0&1&-1&0\\0&4&-4&0}\myvec{x_1\\x_2\\x_3\\x_4}&=\myvec{0\\0\\0\\0\\0}
\end{align}
Using Gaussian Elimination method :
\begin{align}
\xleftrightarrow{R_2 \leftrightarrow R_5}\myvec{1&0&-1&0&:&0\\0&4&-4&0&:&0\\0&2&0&-1&:&0\\0&1&-1&0&:&0\\2&0&0&-1&:&0}\\
\xleftrightarrow{R_5 \leftarrow 2R_1-R_5}\myvec{1&0&-1&0&:&0\\0&4&-4&0&:&0\\0&2&0&-1&:&0\\0&1&-1&0&:&0\\0&0&-2&1&:&0}\\
\xleftrightarrow[R_4 \leftarrow 4R_4-R_2]{R_3 \leftarrow 2R_3-R_2}\myvec{1&0&-1&0&:&0\\0&4&-4&0&:&0\\0&0&4&-2&:&0\\0&0&0&0&:&0\\0&0&-2&1&:&0}\\
\xleftrightarrow{R_5 \leftrightarrow R_5}\myvec{1&0&-1&0&:&0\\0&4&-4&0&:&0\\0&0&4&-2&:&0\\0&0&-2&1&:&0\\0&0&0&0&:&0}\\
\xleftrightarrow{R_4 \leftarrow 2R_4-R_3}\myvec{1&0&-1&0&:&0\\0&4&-4&0&:&0\\0&0&4&-2&:&0\\0&0&0&0&:&0\\0&0&0&0&:&0}
\end{align}
Clearly the system is linearly dependent. Therefore by fixing the value of $x_4=2$, one of the possible vector $\vec{x}$ is:
\begin{align}
    \vec{x}=\myvec{1\\1\\1\\2}
\end{align}
Hence by putting the values of $x_1,x_2,x_3,x_4$ in equation \eqref{eq:1} we get our balanced chemical equation as followes :
\begin{align}
BaCl_2 + K_2SO_4 \rightarrow BaSO_4 + 2KCl
\end{align}

\end{document}

