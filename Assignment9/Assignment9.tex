\documentclass[journal,12pt,twocolumn]{IEEEtran}
\usepackage[compact]{titlesec}
\usepackage{setspace}
\usepackage{gensymb}
\singlespacing
\usepackage[cmex10]{amsmath}
\usepackage{amsthm}
\usepackage{mathrsfs}
\usepackage{txfonts}
\usepackage{stfloats}
\usepackage{bm}
\usepackage{cite}
\usepackage{cases}
\usepackage{subfig}
\usepackage{longtable}
\usepackage{multirow}
\usepackage{enumitem}
\usepackage{mathtools}
\usepackage{steinmetz}
\usepackage{tikz}
\usepackage{circuitikz}
\usepackage{verbatim}
\usepackage{tfrupee}
\usepackage[breaklinks=true]{hyperref}
\usepackage{tkz-euclide}

\usetikzlibrary{calc,math}
\usepackage{listings}
    \usepackage{color}                                            %%
    \usepackage{array}                                            %%
    \usepackage{longtable}                                        %%
    \usepackage{calc}                                             %%
    \usepackage{multirow}                                         %%
    \usepackage{hhline}                                           %%
    \usepackage{ifthen}                                           %%

    \usepackage{lscape}     
\usepackage{multicol}
\usepackage{chngcntr}

\DeclareMathOperator*{\Res}{Res}
\renewcommand\thesection{\arabic{section}}
\renewcommand\thesubsection{\thesection.\arabic{subsection}}
\renewcommand\thesubsubsection{\thesubsection.\arabic{subsubsection}}

\renewcommand\thesectiondis{\arabic{section}}
\renewcommand\thesubsectiondis{\thesectiondis.\arabic{subsection}}
\renewcommand\thesubsubsectiondis{\thesubsectiondis.\arabic{subsubsection}}

\hyphenation{op-tical net-works semi-conduc-tor}
\def\inputGnumericTable{}                                 %%

\lstset{
frame=single, 
breaklines=true,
columns=fullflexible
}

\begin{document}

\newtheorem{theorem}{Theorem}[section]
\newtheorem{problem}{Problem}
\newtheorem{proposition}{Proposition}[section]
\newtheorem{lemma}{Lemma}[section]
\newtheorem{corollary}[theorem]{Corollary}
\newtheorem{example}{Example}[section]
\newtheorem{definition}[problem]{Definition}
\newcommand{\BEQA}{\begin{eqnarray}}
\newcommand{\EEQA}{\end{eqnarray}}
\newcommand{\define}{\stackrel{\triangle}{=}}
\bibliographystyle{IEEEtran}

\providecommand{\mbf}{\mathbf}
\providecommand{\pr}[1]{\ensuremath{\Pr\left(#1\right)}}
\providecommand{\qfunc}[1]{\ensuremath{Q\left(#1\right)}}
\providecommand{\sbrak}[1]{\ensuremath{{}\left[#1\right]}}
\providecommand{\lsbrak}[1]{\ensuremath{{}\left[#1\right.}}
\providecommand{\rsbrak}[1]{\ensuremath{{}\left.#1\right]}}
\providecommand{\brak}[1]{\ensuremath{\left(#1\right)}}
\providecommand{\lbrak}[1]{\ensuremath{\left(#1\right.}}
\providecommand{\rbrak}[1]{\ensuremath{\left.#1\right)}}
\providecommand{\cbrak}[1]{\ensuremath{\left\{#1\right\}}}
\providecommand{\lcbrak}[1]{\ensuremath{\left\{#1\right.}}
\providecommand{\rcbrak}[1]{\ensuremath{\left.#1\right\}}}
\theoremstyle{remark}
\newtheorem{rem}{Remark}
\newcommand{\sgn}{\mathop{\mathrm{sgn}}}
\providecommand{\abs}[1]{\left\vert#1\right\vert}
\providecommand{\res}[1]{\Res\displaylimits_{#1}} 
\providecommand{\norm}[1]{\left\lVert#1\right\rVert}

\providecommand{\mtx}[1]{\mathbf{#1}}
\providecommand{\mean}[1]{E\left[ #1 \right]}
\providecommand{\fourier}{\overset{\mathcal{F}}{ \rightleftharpoons}}

\providecommand{\system}{\overset{\mathcal{H}}{ \longleftrightarrow}}
\newcommand{\solution}{\noindent \textbf{Solution: }}
\newcommand{\cosec}{\,\text{cosec}\,}
\providecommand{\dec}[2]{\ensuremath{\overset{#1}{\underset{#2}{\gtrless}}}}
\newcommand{\myvec}[1]{\ensuremath{\begin{pmatrix}#1\end{pmatrix}}}
\newcommand{\mydet}[1]{\ensuremath{\begin{vmatrix}#1\end{vmatrix}}}
\numberwithin{equation}{subsection}

\makeatletter
\@addtoreset{figure}{problem}
\makeatother
\let\StandardTheFigure\thefigure
\let\vec\mathbf

\renewcommand{\thefigure}{\theproblem}

\def\putbox#1#2#3{\makebox[0in][l]{\makebox[#1][l]{}\raisebox{\baselineskip}[0in][0in]{\raisebox{#2}[0in][0in]{#3}}}}
     \def\rightbox#1{\makebox[0in][r]{#1}}
     \def\centbox#1{\makebox[0in]{#1}}
     \def\topbox#1{\raisebox{-\baselineskip}[0in][0in]{#1}}
     \def\midbox#1{\raisebox{-0.5\baselineskip}[0in][0in]{#1}}
\vspace{3cm}
\title{Assignment-9}
\author{Vipul Kumar Malik}

\date{\today}

\maketitle
\newpage
\bigskip
\renewcommand{\thefigure}{\theenumi}
\renewcommand{\thetable}{\theenumi}

\begin{abstract}
This document explains the concept of finding the foot of the perpendicular to the line from the given point using Singular Value Decomposition (SVD).
\end{abstract}
Download all python codes from 
\begin{lstlisting}
https://github.com/vipulmalik8569/MT-EE5609
\end{lstlisting}
and latex-tikz codes from 
\begin{lstlisting}
https://github.com/vipulmalik8569/MT-EE5609
\end{lstlisting}
\section{\textbf{Problem}}
Find the foot of the perpendicular to the line from the point $\vec{p}$. 

Equation of the line and point $\vec{p}$ is given as : 
\begin{align}
\frac{3-x}{2}=\frac{y-1}{2}&=\frac{z+6}{-1}=t\label{eq:0}\\
\vec{p}&=\myvec{-2\\3\\8}
\end{align}
\section{\textbf{Solution}}
Vector form of the equation \eqref{eq:0} in terms of slope vector $\vec{m}$ and positional vector $\vec{c}$ is :
\begin{align}
\vec{x}=\vec{m}t+\vec{c}\label{eq:1}
\end{align}
Where,
\begin{align}
    \vec{m}=\myvec{-2\\2\\-1}\quad\vec{c}=\myvec{3\\1\\-6}
\end{align}
After the Affine transformation of
\begin{align}
    \vec{x}=\vec{I}\vec{x^*}+\vec{c}\label{eq:3}
\end{align}
equation \eqref{eq:1} and point $\vec{p}$ will become :
\begin{align}
    \vec{x^*}&=\vec{m}t\label{eq:4}\\
    \vec{p}^*&=\vec{p}-\vec{c}=\myvec{-5\\2\\14}
\end{align}
Now let us solve :
\begin{align}
    \vec{M}\vec{x}=\vec{b}\label{eq:5}
\end{align}
Where,
\begin{align}
    \vec{M}=\vec{m}=\myvec{-2\\2\\-1},\quad\vec{b}=\vec{p^*}=\myvec{-5\\2\\14}
\end{align}
This is clearly the case of Over-determined system of equations.

From the Singular value Decomposition on $\vec{M}$ we get :
\begin{align}
\vec{M} = \vec{U}\vec{S}\vec{V}^T\label{eq:7}
\end{align}
By substituting the value of $\vec{M}$ from equation \eqref{eq:7} to \eqref{eq:5} we get :
\begin{align}
\vec{U}\vec{S}\vec{V}^T\vec{x} &= \vec{b}\\
\vec{x} &= \vec{V}\vec{S}_+\vec{U}^T\vec{b} \label{eq:10}
\end{align}
Where,


$\vec{S}_+$ is the Moore-Pen-rose Pseudo-Inverse of $\vec{S}$.


Columns of $\vec{U}$ are eigenvectors of $\vec{M}\vec{M}^T$.


Columns of $\vec{V}$ are eigenvectors of $\vec{M}^T\vec{M}$.


$\vec{S}$ is diagonal matrix of singular value of eigenvalues of $\vec{M}^T\vec{M}$.


Now the eigenvectors corresponding to $\vec{M}^T\vec{M}$ can be calculated as : 
\begin{align}
\vec{M}^T\vec{M} =  \myvec{-2&2&-1}\myvec{-2\\2\\-1} = \myvec{9}
\end{align}
Eigenvalues corresponding to $\vec{M}^T\vec{M}$  is,
\begin{align}
\mydet{\vec{M}^T\vec{M}-\lambda\vec{I}} &= 0\\
9-\lambda&= 0\\
\lambda &= 9\label{eq:lambda2}
\end{align} 
Hence the Normalised eigenvector corresponding to $\lambda$ is,
\begin{align}
\vec{V}&=\frac{1}{9}\myvec{9}=\myvec{1}
\end{align}
Eigenvectors corresponding to $\vec{M}\vec{M}^T$ can be calculated as : 
\begin{align}
\vec{MM}^T = \myvec{-2\\2\\-1}\myvec{-2&2&-1} \\\implies \myvec{4&-4&2\\-4&4&-2\\2&-2&1}
\end{align}
Eigenvalues corresponding to $\vec{M}\vec{M}^T$  are,
\begin{align}
\mydet{\vec{M}\vec{M}^T-\lambda\vec{I}} &= 0\\
\mydet{4-\lambda&-4&2\\-4&4-\lambda&-2\\2&-2&1-\lambda}&=0\\
\lambda^2(\lambda-9)&=0
\end{align} 
Solving this we get : 
\begin{align}
    \lambda_1=0 \quad \lambda_2=9
\end{align}
Hence the eigenvectors corresponding to $\lambda_1$, $\lambda_2$ and  $\lambda_3$ are as follows :
\begin{align}
\vec{u_1} =\myvec{1\\1\\0},
\vec{u_2} =\myvec{\frac{-1}{2}\\0\\1},
\vec{u_3} = \myvec{2\\-2\\1}
\end{align}
Normalizing the eigenvectors we get,
\begin{align}
\vec{u_1} &= \frac{1}{\sqrt{2}}\myvec{1\\1\\0} = \myvec{\frac{1}{\sqrt{2}}\\[0.2cm]\frac{1}{\sqrt{2}}\\[0.2cm]0}\\
\vec{u_2} &= \frac{\sqrt{5}}{2}\myvec{\frac{-1}{2}\\[0.1cm]0\\1}=\myvec{\frac{-\sqrt{5}}{4}\\[0.1cm]0\\[0.1cm]\frac{\sqrt{5}}{2}}\\
\vec{u_3} &= \frac{1}{3}\myvec{2\\-2\\1}= \myvec{\frac{2}{3}\\[0.1cm]-\frac{2}{3}\\[0.1cm]\frac{1}{3}}\\
\vec{U} &= \myvec{\frac{1}{\sqrt{2}} &\frac{-\sqrt{5}}{4}&\frac{2}{3}\\[0.1cm]\frac{1}{\sqrt{2}}&0&\frac{-2}{3}\\[0.1cm]0&\frac{\sqrt{5}}{2}&\frac{1}{3}}
\end{align} 
$\vec{S}$ corresponding to eigenvalues $\lambda_1$ and $\lambda_2$ is as follows : 
\begin{align}
\vec{S} = \myvec{3\\0\\0}
\end{align}
Moore-Penrose Pseudo inverse of $\vec{S}$ is given by,
\begin{align}
\vec{S}_+ = \myvec{\frac{1}{3}&0&0}
\end{align}
Hence the singular value decomposition of $\vec{M}$ is as follows : 
\begin{align}
\vec{M} = \myvec{\frac{1}{\sqrt{2}} &\frac{-\sqrt{5}}{4}&\frac{2}{3}\\[0.1cm]\frac{1}{\sqrt{2}}&0&\frac{-2}{3}\\[0.1cm]0&\frac{\sqrt{5}}{2}&\frac{1}{3}}\myvec{3\\0\\0}\myvec{1}
\end{align}
By putting the values of $\vec{V}$ ,  $\vec{S_+}$ , $\vec{U}$ and $\vec{b}$ in \eqref{eq:10} we get : 
\begin{align}
\vec{x}=\myvec{1}\myvec{\frac{1}{3}&0&0}\myvec{\frac{1}{\sqrt{2}} &\frac{-\sqrt{5}}{4}&\frac{2}{3}\\[0.1cm]\frac{1}{\sqrt{2}}&0&\frac{-2}{3}\\[0.1cm]0&\frac{\sqrt{5}}{2}&\frac{1}{3}}^T\myvec{-5\\2\\14}\\
\implies \vec{x} = \myvec{0}\label{eq:31}
\end{align}
On verifying by computing the least square solution we get : 
\begin{align}
\vec{M}^T\vec{Mx} &= \vec{M}^T\vec{b}\\
\myvec{-2&2&-1}\myvec{-2\\2\\-1}\vec{x}&=\myvec{-2&2&-1}\myvec{-5\\2\\14}\\
\myvec{9}\vec{x}&=\myvec{0}\\
\vec{x}&=\myvec{0}\label{eq:35}
\end{align}
From equations \eqref{eq:31} and \eqref{eq:35} we can conclude that the solution is verified.\\\\
As our solution is in affine space. It is the foot of the perpendicular to the line \eqref{eq:4} from the point $\vec{p^*}$.


Therefore by using \eqref{eq:3} we get
\begin{align}
    \vec{x}&=\vec{I}\myvec{0}+\myvec{1}\\
    \vec{x}&=\myvec{1}\label{eq:37}
\end{align}
Hence equation \eqref{eq:37} gives the foot of the perpendicular to the line \eqref{eq:1} from the point $\vec{p}$. 
\end{document}
