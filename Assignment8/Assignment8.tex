\documentclass[journal,12pt,twocolumn]{IEEEtran}
\usepackage[compact]{titlesec}
\usepackage{setspace}
\usepackage{gensymb}
\singlespacing
\usepackage[cmex10]{amsmath}
\usepackage{amsthm}
\usepackage{mathrsfs}
\usepackage{txfonts}
\usepackage{stfloats}
\usepackage{bm}
\usepackage{cite}
\usepackage{cases}
\usepackage{subfig}
\usepackage{longtable}
\usepackage{multirow}
\usepackage{enumitem}
\usepackage{mathtools}
\usepackage{steinmetz}
\usepackage{tikz}
\usepackage{circuitikz}
\usepackage{verbatim}
\usepackage{tfrupee}
\usepackage[breaklinks=true]{hyperref}
\usepackage{tkz-euclide}

\usetikzlibrary{calc,math}
\usepackage{listings}
    \usepackage{color}                                            %%
    \usepackage{array}                                            %%
    \usepackage{longtable}                                        %%
    \usepackage{calc}                                             %%
    \usepackage{multirow}                                         %%
    \usepackage{hhline}                                           %%
    \usepackage{ifthen}                                           %%

    \usepackage{lscape}     
\usepackage{multicol}
\usepackage{chngcntr}

\DeclareMathOperator*{\Res}{Res}
\renewcommand\thesection{\arabic{section}}
\renewcommand\thesubsection{\thesection.\arabic{subsection}}
\renewcommand\thesubsubsection{\thesubsection.\arabic{subsubsection}}

\renewcommand\thesectiondis{\arabic{section}}
\renewcommand\thesubsectiondis{\thesectiondis.\arabic{subsection}}
\renewcommand\thesubsubsectiondis{\thesubsectiondis.\arabic{subsubsection}}

\hyphenation{op-tical net-works semi-conduc-tor}
\def\inputGnumericTable{}                                 %%

\lstset{
frame=single, 
breaklines=true,
columns=fullflexible
}

\begin{document}

\newtheorem{theorem}{Theorem}[section]
\newtheorem{problem}{Problem}
\newtheorem{proposition}{Proposition}[section]
\newtheorem{lemma}{Lemma}[section]
\newtheorem{corollary}[theorem]{Corollary}
\newtheorem{example}{Example}[section]
\newtheorem{definition}[problem]{Definition}
\newcommand{\BEQA}{\begin{eqnarray}}
\newcommand{\EEQA}{\end{eqnarray}}
\newcommand{\define}{\stackrel{\triangle}{=}}
\bibliographystyle{IEEEtran}

\providecommand{\mbf}{\mathbf}
\providecommand{\pr}[1]{\ensuremath{\Pr\left(#1\right)}}
\providecommand{\qfunc}[1]{\ensuremath{Q\left(#1\right)}}
\providecommand{\sbrak}[1]{\ensuremath{{}\left[#1\right]}}
\providecommand{\lsbrak}[1]{\ensuremath{{}\left[#1\right.}}
\providecommand{\rsbrak}[1]{\ensuremath{{}\left.#1\right]}}
\providecommand{\brak}[1]{\ensuremath{\left(#1\right)}}
\providecommand{\lbrak}[1]{\ensuremath{\left(#1\right.}}
\providecommand{\rbrak}[1]{\ensuremath{\left.#1\right)}}
\providecommand{\cbrak}[1]{\ensuremath{\left\{#1\right\}}}
\providecommand{\lcbrak}[1]{\ensuremath{\left\{#1\right.}}
\providecommand{\rcbrak}[1]{\ensuremath{\left.#1\right\}}}
\theoremstyle{remark}
\newtheorem{rem}{Remark}
\newcommand{\sgn}{\mathop{\mathrm{sgn}}}
\providecommand{\abs}[1]{\left\vert#1\right\vert}
\providecommand{\res}[1]{\Res\displaylimits_{#1}} 
\providecommand{\norm}[1]{\left\lVert#1\right\rVert}

\providecommand{\mtx}[1]{\mathbf{#1}}
\providecommand{\mean}[1]{E\left[ #1 \right]}
\providecommand{\fourier}{\overset{\mathcal{F}}{ \rightleftharpoons}}

\providecommand{\system}{\overset{\mathcal{H}}{ \longleftrightarrow}}
\newcommand{\solution}{\noindent \textbf{Solution: }}
\newcommand{\cosec}{\,\text{cosec}\,}
\providecommand{\dec}[2]{\ensuremath{\overset{#1}{\underset{#2}{\gtrless}}}}
\newcommand{\myvec}[1]{\ensuremath{\begin{pmatrix}#1\end{pmatrix}}}
\newcommand{\mydet}[1]{\ensuremath{\begin{vmatrix}#1\end{vmatrix}}}
\numberwithin{equation}{subsection}

\makeatletter
\@addtoreset{figure}{problem}
\makeatother
\let\StandardTheFigure\thefigure
\let\vec\mathbf

\renewcommand{\thefigure}{\theproblem}

\def\putbox#1#2#3{\makebox[0in][l]{\makebox[#1][l]{}\raisebox{\baselineskip}[0in][0in]{\raisebox{#2}[0in][0in]{#3}}}}
     \def\rightbox#1{\makebox[0in][r]{#1}}
     \def\centbox#1{\makebox[0in]{#1}}
     \def\topbox#1{\raisebox{-\baselineskip}[0in][0in]{#1}}
     \def\midbox#1{\raisebox{-0.5\baselineskip}[0in][0in]{#1}}
\vspace{3cm}
\title{Assignment-8}
\author{Vipul Kumar Malik}

\date{\today}

\maketitle
\newpage
\bigskip
\renewcommand{\thefigure}{\theenumi}
\renewcommand{\thetable}{\theenumi}

\begin{abstract}
This document explains the concept of finding the QR decomposition of a $2\times2$ matrix.
\end{abstract}
Download all python codes from 
\begin{lstlisting}
https://github.com/vipulmalik8569/MT-EE5609
\end{lstlisting}

and latex-tikz codes from 
\begin{lstlisting}
https://github.com/vipulmalik8569/MT-EE5609
\end{lstlisting}
\section{\textbf{Problem}}
Find the QR decomposition of the matrix
\begin{align}
    \vec{V}=\myvec{1&\frac{1}{2}\\[0.1cm]\frac{1}{2}&1}\label{eq:0}
\end{align}
\section{\textbf{Solution}}
Any matrix say $\vec{V}$ with linearly independent column vectors can be factorised as : 
\begin{align}
    \vec{V} = \vec{Q}\vec{R}\label{eq:1}
\end{align}
where $\vec{Q}$ is a orthogonal matrix and $\vec{R}$ is a upper triangular matrix with non zero diagonal elements.

From \eqref{eq:0} let $\vec{a}$ and $\vec{b}$ be the column vectors of $\vec{V}$ written as : 
\begin{align}
    \vec{a} = \myvec{1 \\[0.1cm] \frac{1}{2}}\quad \vec{b} = \myvec{\frac{1}{2} \\[0.1cm] 1}\label{eq:2}
\end{align}
\eqref{eq:1} can also be written as,
\begin{align}
    \vec{V}= \myvec{\vec{p_1} & \vec{p_2}}\myvec{u_1 & u_3 \\ 0 & u_2}\label{eq:3}
\end{align}
where,
\begin{align}
    \vec{Q}=\myvec{\vec{p_1} & \vec{p_2}} \quad \vec{R}=\myvec{u_1 & u_3\\0 & u_2}\label{eq:4}
\end{align}
Now,
\begin{align} 
    u_1 &= \norm{a} = \sqrt{1^2+\left(\frac{1}{2}\right)^2} = \frac{\sqrt{5}}{2} \\
    \vec{p_1} &= \frac{\vec{a}}{u_1} = \myvec{\frac{2}{\sqrt{5}} \\[0.2cm] \frac{1}{\sqrt{5}}} \\
    u_3 &= \frac{\vec{p_1}^T\vec{b}}{\norm{\vec{p_1}}^2} = \myvec{\frac{2}{\sqrt{5}} & \frac{1}{\sqrt{5}}}\myvec{\frac{1}{2}\\1} = \frac{2}{\sqrt{5}}
\end{align}
\begin{align}
    \vec{p_2} &= \frac{\vec{b} - u_3 \vec{p_1}}{\norm{\vec{b} - u_3 \vec{p_1}}} = \myvec{\frac{-2}{5}\\[0.2cm] \frac{9}{10}}\\
    u_2 &= {\vec{p_2}^T\vec{b}} = \myvec{\frac{-2}{5}& \frac{9}{10}} \myvec{3 \\ 5} = \frac{7}{10}\label{eq:8}
\end{align}
By putting the values of $\vec{p_1}$, $\vec{p_2}$, $u_1$, $u_2$ and $u_3$ in \eqref{eq:4} we get : 
\begin{align}
    \vec{Q}=\myvec{\frac{2}{\sqrt{5}}&\frac{-2}{5}\\[0.2cm]\frac{1}{\sqrt{5}}&\frac{9}{10}}\quad \vec{R}=\myvec{\frac{\sqrt{5}}{2}&\frac{2}{\sqrt{5}}\\[0.2cm]0&\frac{7}{10}}
\end{align}
As,
\begin{align}
    \vec{Q}^T\vec{Q}=\myvec{\frac{2}{\sqrt{5}}&\frac{-2}{5}\\[0.2cm]\frac{1}{\sqrt{5}}&\frac{9}{10}}^T\myvec{\frac{2}{\sqrt{5}}&\frac{-2}{5}\\[0.2cm]\frac{1}{\sqrt{5}}&\frac{9}{10}}=\myvec{1&0\\0&1}
\end{align}

Shows $\vec{Q}$ is an Orthogonal matrix.\\

Hence using \eqref{eq:3} the QR Decomposition of $\vec{V}$ can be written as :
\begin{align}
\myvec{1&\frac{1}{2}\\\frac{1}{2}&1}=\myvec{\frac{2}{\sqrt{5}}&\frac{-2}{5}\\[0.2cm]\frac{1}{\sqrt{5}}&\frac{9}{10}}\myvec{\frac{\sqrt{5}}{2}&\frac{2}{\sqrt{5}}\\[0.2cm]0&\frac{7}{10}}
\end{align}
\end{document}
